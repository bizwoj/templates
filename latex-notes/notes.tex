\documentclass[a4paper,12pt]{article}
\pdfoutput=1 % if your are submitting a pdflatex (i.e. if you have
             % images in pdf, png or jpg format)

%-------------------------
% Width and height of the text area correpond the a4wide package
\usepackage[a4paper,textwidth=460.72124pt,textheight=621.0pt]{geometry}
\usepackage{cite}
\usepackage{graphicx}
\usepackage[labelfont=bf]{caption}
\usepackage{subcaption}

%-------------------------
\usepackage[T1]{fontenc} 
\usepackage[utf8]{inputenc}

%-------------------------
\usepackage{amssymb}
\usepackage{amsmath}
\usepackage{amsfonts}
\usepackage{mathtools}
\usepackage{etoolbox} % for simple if constructs
\usepackage{booktabs} % nicer tables

%-------------------------
\DeclareMathAlphabet{\mathpzc}{OT1}{pzc}{m}{it}
\numberwithin{equation}{section}

%-------------------------
\usepackage[colorlinks]{hyperref}
\hypersetup{
 citecolor=blue,
 linkcolor=blue,
 urlcolor=blue}

% clever refs
\usepackage[capitalize]{cleveref}

%% % macros
%% \input{macros.tex}


\title{Notes template}
\author{Wojciech Bizon}
\date{\today}

%-------------------------
\begin{document}


%>>> front page
\pagenumbering{roman}
\pagestyle{empty}


%>>> title and authors
\maketitle


%>>> table of contents
{
  \hypersetup{linkcolor=black}
  \tableofcontents
}
\clearpage


%>>> text
\pagestyle{plain}
\pagenumbering{arabic}
\allowdisplaybreaks

%-------------------------
%\input{sec_intro.tex}

%-------------------------
%\input{sec_general.tex}

%-------------------------
%\input{sec_summary.tex}

%-------------------------
\section*{Acknowledgements}
%
This style shows an easy way of keeping notes in \LaTeX. It supports
math as well as citations to interesting articles such as~\cite{Behring:2019oci}.

% -------------------------
\appendix

%-------------------------
\bibliographystyle{utphys}
\bibliography{refs}{}
\end{document}
